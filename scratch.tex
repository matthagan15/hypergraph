

\section{Homology}


\section{Spitballs}
\subsection{Channels}
Take $h$ a hypergraph over vector space $V = 2^{N} = \bigoplus_{k=0}^{|N|} V_k$, so $h: V \to V$. Take two vector spaces $X \subset V$ and $Y \subset V$. We can construct a linear map between these two spaces by restricting the inputs and outputs: $\widetilde{h} = \Pi_Y \circ h \circ i_X$. Need to prove that the projectors and inclusions exist, the inclusion is easy bc $X$ is a subspace, need to do a little more thinking about the projectors. We say that $\widetilde{h}$ is a channel if it acts as a probability isometry. 

this leads to a few questions:
\begin{itemize}
    \item what is the capacity of this channel?
    \item encoding?
    \item entropies?
    \item huh?
\end{itemize}


Think of hypergraphs as linear maps - first as a collection of tensor products and then as a matrix acting on vectors. We can ask what are random walks in this space like and what are quantum walks in this space like? when do different quantum walks yield the same classical walks, after taking amplitude squared?

\section{Questions}
\begin{enumerate} \label{qs}
    \item computing optimal values of functions on hypergraphs, what is complexity? equivalent to hamiltonian? 
    \item ex: how would I compute optimal placement of a dollar general in the US to maximize expected profit per year?
    \item for example a person would be a hyperedge between the 
    \item what maps from hypergraphs are norm preserving?
    \item think of words as hypergraphs among letters? could this learn languages better? It's a random walk.
    \item can you block encode this action onto a quantum device?
    \item what about hypergraph expanders?
    \item How to use duality between edges and vertices? can this let us take inner products?
    \item computing homology of hypergraph? relation to simplial homologies? khovanov homology? 
    \item One question I have is how to define a walk on it?
    
\end{enumerate}


\section{scratch}

Currently a hypergraph $\eta$ is defined as a collection of the following objects:
\begin{enumerate}
    \item an integer id,
    \item a set $V$ of vertices that supports $\bigo{1}$ access to check if a vertex id is present.
    \item a set $E \coloneqq \set{(v_1, \ldots, v_k)}$ of hyperedges. We may use weighted hypergraphs later. 
    \item a function $\mu : \set{1, 2, \ldots, |V|} \to 2^V$, where $\mu(k) := \set{e \in E \big| |e| = k}$. This can be thought of as the level sets of the hypergraph in terms of the edges with a certain size, if an edge represents a neighborhood this is an open set of a certain measure(?). 
    \item fermions correspond to isometric hypergraphs with homology, bosons do not?
\end{enumerate}
In this case could a hypergraph be considered as a sub-topology(?) of the discrete topology? Also hypergraphs can be viewed as slight generalizations of simplicial complexes as a given edge (aka an $n$-complex) does not need to contain all sub-edges whereas a simplex face needs to contain all sub-faces.

we can also define boundary operators on it, they should replicate the simplicial boundary operator when the sub-edge restriction is added.

So far there seems to be two main methods of traversing a hypergraph that I can think of: 1. A boundary style traversal where a group of vertices is mapped to its boundary and then that is mapped to it's dual-boundary. repeating this, interspersed with boundary or dual-boundary operators will add/delete vertices and allow one to traverse and map a set of vertices to another set. This is called a boundary walk, 2. Consider an undirected hyperedge as containing all possible \emph{directed} hyperedges. For example, the undirected hyperedge $\set{a,b,c}$ would contain the maps $a \mapsto \set{b,c}$, $b \mapsto \set{a,c}$, and so on, also mapping all two vertex sets to the one vertex remainder. If we denote $V^{\otimes k}$ as the set containing $k$ copies of $V$, then we can think of a directed hyperedge as a map $e: V^{\otimes n} \to V^{\otimes m}$. Tensor product may not be the correct notion, but it may. Because this maps an edge set to it's complement within the hyperedge set, this will be called a complement walk. 


Thinking of hypergraph $h$ as a linear map on the vector space $V = \bigoplus_{k=0}^{n}$, where $V$ has $n$ vertices. We can define $(i,j,k)$ clustering quantitatively as the entropy from starting at dimension $i$, taking $j$ steps in the hypergraph, and then project to dimension $k$. Or can define clustering as the probability that you start at a random vertex, take a few steps, and then end up in a certain group.
